\thispagestyle{empty}\section*{\ifthenelse{\boolean{english}}{Abstract}{Zusammenfassung}}
Workload mobility can address a multitude of challenges that typically arise in fog computing, especially in the context of IoT applications. Addressing this issue, this work proposes an approach that allows for consistent migration of stateful stream processing operators in the fog. The two sub-problems migration of a correctly running operator and migration of an interrupted operator are handled separately and are addressed with corresponding procedures. These procedures are then implemented into Apache StreamPipes as a proof of concept. The evaluation of this proof of concept shows that the proposed procedures perform the migration consistently and with a low operator downtime.


%An dieser Stelle erfolgt eine knappe Zusammenfassung der vorliegenden Arbeit ([engl.] Abstract), die maximal ca. 200 Worte umfassen sollte. Der Sinn und Zweck dieser Zusammenfassung liegt darin, einem interessierten Leser die Entscheidung zu erleichtern, die vorliegende Arbeit überhaupt zu lesen bzw. vor dem Lesen der Arbeit erst einmal in Erfahrung zu bringen, worum es dabei geht. Also eine knappe, motivierende Hinführung zum Problem und wie Sie es gelöst haben.

%Wenn Sie eine Zusammenfassung schreiben, bedenken Sie, dass diese oft auch alleine publiziert wird, d.h. sie sollte unabhängig vom nachfolgenden explizit dargestellten Inhalt der Arbeit für den Leser verständlich sein. Daher ist es immer sinnvoll, diese Zusammenfassung erst ganz am Ende zu schreiben, wenn Sie die eigentliche Arbeit bereits abgeschlossen haben.
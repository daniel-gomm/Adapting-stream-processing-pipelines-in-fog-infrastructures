\thispagestyle{empty}\section*{\ifthenelse{\boolean{english}}{Abstract}{Zusammenfassung}}

As the number of \acrfull{iot} devices continues growing, means of effectively deploying \gls{iot} applications become increasingly important. Fog computing can provide an appropriate platform for many of these platforms. A commonly deployed application in such a setting is stream processing. This thesis thus addresses the issue of adapting stream processing pipelines while they are running. Stream processing pipelines perform computations through a sequence of Pipeline Elements (\acrshort{pe}s). These \acrshort{pe}s can be deployed across different fog nodes. Adapting a stream processing pipeline thus refers to the relocation (migration) of \acrshort{pe}s between fog nodes. In the therefore developed conceptual framework, the two sub-problems migration of a correctly running \acrshort{pe} and migration of an interrupted \acrshort{pe} are handled separately and are addressed with respective procedures. These procedures were implemented into Apache StreamPipes (incubating) as a proof of concept. The evaluation of this proof of concept shows that the proposed procedures perform the migration consistently and with a low \acrshort{pe} downtime.




%Workload mobility can address a multitude of challenges that typically arise in fog computing, especially in the context of IoT applications. Addressing this issue, this work proposes an approach that allows for consistent migration of stateful stream processing \gls{pe}s in the fog. The two sub-problems migration of a correctly running \gls{pe} and migration of an interrupted \gls{pe} are handled separately and are addressed with corresponding procedures. These procedures were implemented into Apache StreamPipes (incubating) as a proof of concept. The evaluation of this proof of concept shows that the proposed procedures perform the migration consistently and with a low \gls{pe} downtime.


%An dieser Stelle erfolgt eine knappe Zusammenfassung der vorliegenden Arbeit ([engl.] Abstract), die maximal ca. 200 Worte umfassen sollte. Der Sinn und Zweck dieser Zusammenfassung liegt darin, einem interessierten Leser die Entscheidung zu erleichtern, die vorliegende Arbeit überhaupt zu lesen bzw. vor dem Lesen der Arbeit erst einmal in Erfahrung zu bringen, worum es dabei geht. Also eine knappe, motivierende Hinführung zum Problem und wie Sie es gelöst haben.

%Wenn Sie eine Zusammenfassung schreiben, bedenken Sie, dass diese oft auch alleine publiziert wird, d.h. sie sollte unabhängig vom nachfolgenden explizit dargestellten Inhalt der Arbeit für den Leser verständlich sein. Daher ist es immer sinnvoll, diese Zusammenfassung erst ganz am Ende zu schreiben, wenn Sie die eigentliche Arbeit bereits abgeschlossen haben.
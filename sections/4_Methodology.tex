\section{Methodology}
\label{lMethodology}

In this chapter, a conceptual framework for the migration of a \gls{pe} is proposed. The context of the migration is assessed first. Then the underlying requirements are outlined. After that, a conceptual framework satisfying these requirements is developed. Distinct approaches are proposed for the \textit{\acrshort{pe} live migration} and the \textit{\acrshort{pe} restoration}.\par

\begin{figure}[!ht]
\graphicspath{{./figures/code/}}
\includesvg[width=\textwidth]{figures/visualizations/MigrationContext_1500.svg}
\caption{Context of the migration procedure}
\label{fMigrationContext}
\end{figure}

Figure \ref{fMigrationContext} depicts the context of the migration procedure. The migration can either be initiated to react to situational factors, which can be constantly monitored or on demand to react to influenceable factors. Table \ref{tExamplesFactors} shows examples for these factors. Before moving the application away from the origin node, a target node must be selected. The optimization of this selection process is the subject of many publications regarding application migration in the fog \cite{Goncalves.2018, Yao.2015, Puliafito.2018, Saurez.2016}, but it is not further discussed in this thesis. Instead, the focus is on the next step, the transfer of the \gls{pe} between fog nodes. This relocation process has the goal of moving a \gls{pe} from its \textit{origin node} to a specific \textit{target node}. Subjects of the relocation are orderly \textit{running \gls{pe}s}, which must be stopped on the origin node and resumed on the target node and so-called \textit{interrupted \gls{pe}s}. \gls{pe}s are interrupted when they have become unavailable, for example, due to network outages, network instability, or because they have encountered faults while processing events. These \gls{pe}s need to be removed from the origin node and restored on the target node. The migration is finalized by acknowledging its success, after the \gls{pe} is relocated.\\
In the thesis, the relocation process in step 4 is referred to as migration.\par


\begin{table}[H]
    \caption{Examples for situations in which migration is beneficial}
    \label{tExamplesFactors}
    \centering
    \begin{tabular}{|p{3cm}|p{4cm}|p{8cm}|}
    \hline
     & \textbf{Factor} & \textbf{Need for migration}\\ 
     \hline
     \multirow{4}{4em}{Situational factors}&Resource load on node&Prevent resource overload\\
     \cline{2-3}
     &Position of node&Reduce latency by computing on nodes in the close vicinity\\
     \cline{2-3}
     &Node availability&Applications need to be redeployed if the origin node has become unavailable\\
     \cline{2-3}
     &Remaining battery charge&Reduce load on low battery devices to reduce battery drain\\
     \hline
     \multirow{2}{4em}{Influenceable factors}&Scheduled maintenance of nodes&Maintenance of node leads to unavailability of node for the application\\
     \cline{2-3}
     &Planned usage of nodes&If node is scheduled for a different purpose in the future it will not be available for the application any longer\\
     \hline
    \end{tabular}
\end{table}


\subsection{Requirements}
\label{lRequirements}
As a guideline for developing and evaluating the conceptual migration framework, requirements on this framework are outlined in this section.\\
The requirements are derived from the research questions and further expand on them. Requirements can be categorized into functional and non-functional requirements. Functional requirements describe what the proposed conceptual framework should deliver. Non-functional requirements describe how the proposed conceptual framework should deliver its functionality.\par


\makeatletter
\newcommand{\myitem}[1]{%
\item[#1]\protected@edef\@currentlabel{#1}%
}
\makeatother


\textbf{Functional Requirements}
\begin{enumerate}[wide=0em, labelindent=0.25cm, leftmargin=4.25cm, labelwidth=4cm, labelsep=0em]
    \myitem{\textit{Basic Migration}}\label{rBasicMigration} The conceptual framework shall allow stateful and stateless \gls{pe}s to be migrated from an origin node to a target node.
    \myitem{\textit{Recovery}}\label{rRecovery} The conceptual framework shall support the migration of running \gls{pe}s and interrupted \gls{pe}s.
    \myitem{\textit{Problem Handling}}\label{rProblemHandling} Possible problems shall be handled to facilitate continued processing.
\end{enumerate}

\textbf{Non-Functional Requirements}
\begin{enumerate}[wide=0em, labelindent=0.25cm, leftmargin=4.25cm, labelwidth=4cm, labelsep=0em]
    \myitem{\textit{Consistency}}\label{rConsistency} The resulting output-stream of the migrated \gls{pe} shall be consistent with the output-stream that would have been generated if no migration had happened. The optimal scenario is one in which no events are lost, and no events are processed a second time, a so-called exactly once semantic \cite{Huang.2001}.
    \myitem{\textit{Downtime}}\label{rDowntime} The time in which event processing is disrupted (referred to as downtime) shall be minimized.
    \myitem{\textit{Development}}\label{rDevelopment} An additional complication in \gls{pe} development due to the introduction of \gls{pe} migration shall be minimized.
    \myitem{\textit{Fog Compatibility}}\label{rFogCompatibility} The application shall be deployable in the fog, which means that the fog-specific requirements resource constraints \cite{Yousefpour.2019} and varying networking conditions \cite{Puliafito.2018} have to be considered.
\end{enumerate}

The proposed requirements define the nature of an optimal migration procedure. The goal of the conceptual framework developed in this thesis is to satisfy them as much as possible.\\
As mentioned these requirements are derived from the research questions. The \ref{rBasicMigration} and the \ref{rRecovery} requirements are derived from \ref{rqMigration}. The \ref{rConsistency} and the \ref{rProblemHandling} requirements are extracted from \ref{rqConsistency}. As it describes the overall setting considered in this thesis, the \ref{rFogCompatibility} requirement was derived from both requirements. Lastly, the \ref{rDowntime} and the \ref{rDevelopment} requirements expand on the research questions.\par
Looking at the requirements, a particularly important implication for the conceptual framework can be derived from the \ref{rRecovery} requirement: The developed conceptual framework should allow the migration of interrupted \gls{pe}s, meaning \gls{pe}s that encountered faults or are unavailable. Migrating these \gls{pe}s means restoring their state, which is only possible if state checkpointing measures are implemented. These measures need to provide the possibility to restore a \gls{pe}'s state even if the origin node is unreachable (e.g., in the case of a power outage).

\subsection{Conceptual Migration Framework}
\label{lMigrationConcept}
Using the requirements, a conceptual framework for the migration is developed in this section. The conceptual framework is proposed for a fog computing application that consists of a centralized orchestrator and an undefined number of \gls{pe}s deployed on computing nodes in the fog. The centralized orchestrator controls the deployment and the overall life-cycle of all pipelines in the application.\par %In the following considerations, the migrated application is one of these \gls{pe}s.
As a first step, the different existing solutions presented in chapter \ref{lRelatedWork} have to be assessed regarding which category of approaches best fits the requirements at hand. Commonly proposed approaches are \gls{vm} live migration, container migration using checkpoint/restore functionalities, and explicit state management at the application level. This assessment is shown in Table \ref{tTechnologyComparison}.\\

\begin{table}[!ht]
    \caption{Comparison of migration approaches in related works in regards to their fit to the requirements}
    \label{tTechnologyComparison}
    \begin{tabular}{|p{4.5cm}|P{3.5cm}|P{3.5cm}|P{3.5cm}|}
    \hline
     \textbf{Requirement} & \textbf{Virtual Machine Live Migration} & \textbf{Checkpoint/ Restore Container Migration} & \textbf{Explicit State Management}\\ 
     \hline
     \ref{rBasicMigration} & \checkmark & \checkmark & \checkmark\\
     \hline
     \ref{rRecovery} & \ding{55} & \ding{55} & \checkmark\\
     \hline
     \ref{rProblemHandling}& \checkmark & \checkmark & \checkmark\\
     \hline
     \ref{rConsistency}& \textasciitilde & \textasciitilde & \checkmark\\
     \hline
     \ref{rDevelopment} & \checkmark & \checkmark& \textasciitilde\\
     \hline
     \ref{rDowntime} & \checkmark & \checkmark & \checkmark\\
     \hline
     \ref{rFogCompatibility} & \textasciitilde & \checkmark & \checkmark\\
     \hline
    \end{tabular}
\end{table}

%Unterscheiden nach: 
%\checkmark: achievable with no/kleine Anpassungen
%\textasciitilde: achievable with considerable effort
%\ding{55}: difficult to achieve, even with considerable effort

When comparing the approaches in Table \ref{tTechnologyComparison}, the similarities between the \gls{vm} live migration, and the checkpoint/restore container migration approaches become visible. In terms of their capabilities to fulfill the requirements, the only difference is that \gls{vm} live migration approaches are less appropriate in terms of the \ref{rFogCompatibility}. This is apparent since containerization is the better-suited technology for virtualization on low resource computing devices \cite{Jalali.2016}, which is also why deploying fog computing applications in \gls{vm}s has been becoming less popular in recent years \cite{Puliafito.2018}. Because of this reasoning, the \gls{vm} level approaches are dismissed in favor of the container checkpoint/restore based approaches.\\
Excluding the \gls{vm} live migration leaves explicit state management and the checkpoint/restore container approaches to be compared further. Both classes of approaches fundamentally address the \ref{rBasicMigration} requirement, both can adopt measures for \ref{rProblemHandling}, as wells as measures to minimize the \ref{rDowntime}, and both can be provide \ref{rFogCompatibility}.\\ 
The main differences of their capabilities therefore lie in their fulfillment of the \ref{rDevelopment}, \ref{rRecovery} and \ref{rConsistency} requirements. While the checkpoint/restore container migration approaches migrate the whole container's state, with explicit state management, only the state of particular \gls{pe}s is migrated. The container's state is acquired by checkpointing the whole container, which does not require any additional actions by a \gls{pe} developer. In explicit state management, developers need to expose the \gls{pe}'s state, most commonly abstracted as operator state. Possibilities for exposing the state include leaving requiring developers to implement getter and setter functions for the state \cite{CastroFernandez.2013, Saurez.2016} and requiring source code annotations used to mark the \gls{pe}'s state \cite{Gibson.2014}. This means that even though the additionally needed development effort for \gls{pe}s can be minimized, the explicit state management approaches fulfill the \ref{rDevelopment} requirement to a lesser extent than the checkpoint/restore container approached.\\
When using explicit state management at the application level, the state can be abstracted as operator state. This provides an explicit representation of the buffer state. A correct buffer state is essential to enable \ref{rRecovery} since the correct processing state can only be restored when its corresponding position in the event stream is known. This is problematic when looking at the checkpoint/restore container approaches since the container checkpoint has no clear representation for the buffer state. While it may be possible to address this issue by adding either constant monitoring of the \gls{pe}'s porcessing position in the event stream or by explicitly informing a \gls{pe} about the migration, this added overhead would most probably come at the cost of impairments to the processing performance. This makes explicit state management the more appropriate approach in regards to the \ref{rRecovery} requirement.\\
Similarly, a missing representation for a buffer state hinders the checkpoint/restore container approaches in terms of the \ref{rConsistency} requirement. The \ref{rConsistency} outlines an exactly-once semantic as the optimal outcome. To achieve this, the buffer state information is needed to guarantee that no event is processed twice or not processed at all. If a checkpoint/restore container approach would be followed, this would have to be addressed in-depth to fulfill the \ref{rConsistency} requirement better.\\
Summarizing, the checkpoint/restore container approaches achieve the ability to migrate an entire container without additional development effort at the cost of less control over the migrated contents and the context of the containers state. This makes providing measures for fault-tolerance difficult, which is required by the \ref{rRecovery} requirement. Additionally to these findings, the checkpoint/restore container approaches introduce a technology dependency between the containerization technology and the checkpointing tool, which can be problematic. This is shown by the fact that there are compatibility problems between some containerization technologies, like Docker, and the checkpointing tool \gls{criu} \cite{Brogi.2018, .27032020}.\\
Due to the overall better fit to the requirements and to stream processing in the fog, the developed conceptual framework explores explicit state management at the application level further.\par


%Looking at the comparison in Table \ref{tTechnologyComparison} it becomes clear that the container level migration approaches achieve the ability to migrate an entire container at the cost of less control over the migrated contents and the context of the containers state. This makes providing measures for fault-tolerance difficult, which is required by the \ref{rRecovery} requirement. Another conflict arises with the \ref{rConsistency} requirement, since application-level mechanisms are needed to retrieve buffer-state information. Without the buffer-state information a consistent migration result cannot be guaranteed. Furthermore the container level approaches introduce a technology dependency to the containerization technology and the checkpointing tool, which can be problematic. This is shown by the fact that there are compatibility problems between some containerization technologies, like Docker, and the checkpointing tool \gls{criu}\cite{Brogi.2018, .27032020}.\\
%In contrast to this, the application level approaches mostly abstract the state as \gls{pe} state, which naturally has a representation for the buffer state. In these approaches the processing state is left to developers to expose through getter and setter functions \cite{CastroFernandez.2013, Saurez.2016} or through source code annotations\cite{Gibson.2014}. On the one hand this introduces inconveniences for developers, on the other hand this additional effort also yields some benefits, like the possibility to minimize the size of the serialized processing state for the migration or to minimize its serialization time, addressing the minimization of the \ref{rDowntime}. Providing easy access to buffer, routing, and processing state information application level approaches also provide accessible possibilities to realize state checkpointing.\\
%Because of the outlined good fit of the application level approaches to the use case of stream processing \gls{pe}s in the fog this direction is further explored with the developed concept.\par

According to Ding et al. \cite{Ding.15012015}, the two most important, but often neglected, questions when designing a mechanism for state migration are how to migrate and what to migrate.\\
In the development of the conceptual migration framework, these two questions are therefore addressed. Since the migration process is dependent on the migration's contents, the question of what to migrate is answered first.\par

\subsubsection{Operator State}
\label{lOperatorState}
Like most other application-level approaches, the state is abstracted as operator state, due to the benefits mentioned above. As described in \ref{lStreamProcessing}, so-called stateless \gls{pe}s possess routing and buffer state, which means that they can be treated like stateful \gls{pe}s with no processing state.\\
On the application level, the components of the operator state have to be addressed individually:\par

\textbf{Routing State}\\
The centralized orchestrator handles the routing state. Since the centralized orchestrator keeps track of the information of all deployed pipelines, it also possesses the information about the routing state. During migration, the orchestrator provides the required routing state and adjusts routing accordingly.\par

\textbf{Processing State}\\
The \gls{pe} possesses the processing state. It can be extracted from the \gls{pe} and serialized for the migration. Additionally, the \gls{pe} offers an interface that allows inserting a serialized processing state.\par

\textbf{Buffer State}\\
Handling the buffer state consists of handling the \gls{pe}'s position in the event stream and handling the output buffer. The \gls{pe} keeps track of which events have been processed and which have not. This is referred to as the \gls{pe}'s position in the event stream, which is easily accessible. The output buffer has to be handled depending on its implementation. A possible way to handle the output buffer is by using a messaging service between \gls{pe}s that possesses inbuilt capabilities to replay events. Another possibility is to keep output events stored on the upstream node and replay events from there, as proposed by Fernandez et al. \cite{CastroFernandez.2013}.\par

\subsubsection{Migration Procedure}
\label{lMigrationProceedure}

As laid out before, two migration situations are distinguishable:
\begin{enumerate}
    \item \label{cRunning} A correctly running \gls{pe} is migrated from the origin node to the target node.
    \item \label{cInterrupted} A \gls{pe} that is interrupted (e.g., origin node lost network connection; power outage at origin node, etc.) is migrated to the target node.
\end{enumerate}

In the first situation, it is possible to get the current state of the \gls{pe} on the origin node and to migrate this state. This is not possible in the second situation since there is no available \gls{pe} whose state could be transferred. Instead, a checkpoint of a past version of the \gls{pe}'s state is needed to recreate the current operator state on the target node and to start processing afterward. Therefore, this situation requires a mechanism to get operator state checkpoints, which are available even when the origin node is no longer reachable, as well as a mechanism to restore from a past checkpoint.\par

To account for the differences in these situations, two distinct approaches are proposed. The concept for situation \ref{cRunning} is referred to as \textit{\textit{\acrshort{pe} live migration}} and the concept for situation \ref{cInterrupted} as \textit{\textit{\acrshort{pe} restoration}}. To enable the \textit{\acrshort{pe} restoration} \acrfull{dlac} is proposed to provide the needed checkpointing capabilities.\par

\textbf{\acrlong{pe} Live Migration}\\
The centralized orchestrator coordinates the \textit{\acrshort{pe} live migration}. It handles the deployment of all pipelines and thus possesses the information corresponding to the affected \gls{pe}s. The procedure is depicted in Figure \ref{fOperatorStateMigration}. The displayed activity of the \gls{pe}s in the chart represents the event processing activity.\\
After the migration procedure is triggered, the centralized orchestrator pauses processing on the origin node and fetches its current operator state. The \gls{pe} on the target node is invoked with the received operator state. The target node uses the received operator state to set its processing state and buffer state. After that, processing is started. The \gls{pe} on the target node then informs the centralized orchestrator about whether the invocation was successful or unsuccessful. If it was successful, the paused \gls{pe} on the origin node is discarded. If it was unsuccessful, the \gls{pe} on the target node is discarded, and the \gls{pe} on the origin node is resumed. This distinction protects from problems while inserting the operator state on the target node and other invocation related problems, therefore addressing the \ref{rProblemHandling} requirement.\\
This procedure requires \gls{pe}s to offer an interface over which event processing can be paused and resumed, and over which the operator state is accessible.\par

\begin{figure}[H]
\centering
\graphicspath{{./figures/code/}}
\includesvg[width=13cm]{figures/uml/operator_state_migration_1300.svg}
\caption{Procedure for the \textit{\acrshort{pe} live migration}}
\label{fOperatorStateMigration}
\end{figure}


\textbf{Dual Level Asynchronous Checkpointing}\\
As mentioned before, it is necessary to integrate measures for state checkpointing to fulfill the \ref{rRecovery} requirement. These measures must allow the state recreation even when the node with the faulty \gls{pe} is unreachable. To achieve this \gls{dlac} is proposed.\par
As a measure for fault-tolerance, a set of databases is deployed. Figure \ref{fDLAC} depicts the setup for \gls{dlac}. The setup consists of a database at every node that hosts \gls{pe}s (\gls{pe}-level) and a centralized database that sits on the same centralized node as the centralized orchestrator (central-level). On the \gls{pe}-level checkpoints of all running \gls{pe}s are acquired asynchronously and stored in the database. To achieve a consistent local checkpoint, the state acquisition requires a \gls{pe} to pause processing events while its processing and buffer state information is gathered and returned. Additionally, checkpoints of all running \gls{pe}s are retrieved at the central-level. For the central-level the checkpoints are obtained by contacting a \gls{pe}-level interface that returns the most recently acquired checkpoints from the nodes database.\par
\begin{figure}[H]
\graphicspath{{./figures/code/}}
\includesvg[width=\textwidth]{figures/visualizations/DLAC_1500.svg}
\caption{Setup for Dual Level Asynchronous Checkpointing}
\label{fDLAC}
\end{figure}
The deployed tactic for checkpointing is asynchronous independent checkpointing. This tactic is chosen since (a-)synchronous global checkpointing does not offer any benefits for the use case at hand, while imposing a higher complexity (e.g., Chandy-Lamport algorithm) or performance impairments (e.g., "stop-the-world" approaches). Instead, the \gls{pe}s are checkpointed independently, which is sufficient since the state is not used to recreate the whole pipeline from a specific point in time, but rather to recreate single \gls{pe}s.\\
While the use case of \textit{\acrshort{pe} restoration} solely relies on the centrally available state checkpoints, the additional co-location of operator state checkpoints with their respective \gls{pe} allows for a local redeployment in the case of a fault at the \gls{pe}. This local redeployment falls outside the scope of this thesis but should be addressed by future work.\\
Additionally, the event processing impairment is minimized by offering an interface, which provides the most recent checkpointed state, instead of offering the current operator state, whose acquisition would require event processing to be temporarily paused. This means that collecting checkpoints at the central-level does not influence event processing directly.\par


\textbf{\acrlong{pe} Restoration}\\
Figure \ref{fTimingOSR} shows the situation encountered when the \textit{\acrshort{pe} restoration} is triggered. Three points in time are particularly important:

\begin{enumerate}
    \item \label{tCheckpoint}The time when the last checkpoint of the \gls{pe} on the origin node was recorded before it was interrupted.
    \item \label{tInterrupted}The point in time when the \gls{pe} on the origin node was interrupted.
    \item \label{tRestore} The timing at which the \gls{pe} is restored on the target node.
\end{enumerate}

The operator state's available checkpoint is most likely deprecated at the time the state restoration is triggered since events might have been processed after the last checkpoint was recorded and synchronized with the centralized database. To recreate the state the \gls{pe} had at \ref{tInterrupted}, the events that were processed between \ref{tCheckpoint} and \ref{tInterrupted} have to be reprocessed. Reprocessing refers to processing events in the \gls{pe} without outputting events to an event stream. This recreates the processing state at \ref{tInterrupted}. After that, the events that have accumulated between \ref{tInterrupted} and \ref{tRestore} are processed before the regular processing of events can begin. The events between \ref{tInterrupted} and \ref{tRestore} are referred to as intermediate events.\\

\begin{figure}[!ht]
\graphicspath{{./figures/code/}}
\includesvg[width=\textwidth]{figures/visualizations/stateRestorationTiming_1500.svg}
\caption{Context of the \textit{\acrshort{pe} restoration}}
\label{fTimingOSR}
\end{figure}

The procedure for the \textit{\acrshort{pe} restoration} is depicted in Figure \ref{fOSR}. After the \textit{\acrshort{pe} restoration} is triggered for a specific interrupted \gls{pe} and a target node, the centralized orchestrator fetches the latest available checkpoint of the interrupted \gls{pe}'s state from the centralized database. This checkpoint is then used to invoke a new \gls{pe} on the target node. The newly invoked \gls{pe} then reprocesses events, as described before, then processes the intermediate events until it catches up to the latest events and afterward starts processing incoming events regularly. As the last step, the newly invoked \gls{pe} informs the centralized orchestrator about its status. This information can be used to decide whether additional actions, like retrying the \textit{\acrshort{pe} restoration}, are needed.\par

\begin{figure}[!ht]
\graphicspath{{./figures/code/}}
\includesvg[width=\textwidth]{figures/uml/PERestoration_1500.svg}
\caption{Procedure for the \textit{\acrshort{pe} restoration}}
\label{fOSR}
\end{figure}

\subsubsection{Assessment of the Conceptual Migration Framework}
\label{lEvaluationOfMigrationConcept}
To close this chapter, the fit between the conceptual framework and the requirements is discussed.\\
The \ref{rBasicMigration} requirement and the \ref{rRecovery} requirement are the basis of the conceptual framework. Hence, these requirements are addressed throughout the conceptual framework.\\
The \textit{\acrshort{pe} live migration} focuses on the \ref{rProblemHandling} requirement by keeping the \gls{pe} on the origin node paused until a successful migration is confirmed. Otherwise, the paused \gls{pe} could be resumed, and the migration could be retried at a later point in time or with another target node. The \textit{\acrshort{pe} restoration} cannot provide this safety measure since the \gls{pe} on the origin node is not available in this case. A rudimentary measure is the return of the restoration status, which allows a situation-specific handling. For example, the \textit{\acrshort{pe} restoration} could be retried with another target node or from an older checkpoint.\\
The consistency of the resulting \gls{pe} output (\ref{rConsistency} requirement) is addressed by including explicit information about the position in the event stream, from which processing should be picked up through the buffer state information. This information is mandatory for a consistent \textit{\acrshort{pe} live migration}, as well as for a consistent \textit{\acrshort{pe} restoration}.\\
The minimization of the \ref{rDowntime} is considered in the \textit{\acrshort{pe} live migration} as event processing is only paused for the duration of the state transferal and the invocation of the new \gls{pe}.\\
The \ref{rDevelopment} and \ref{rFogCompatibility} requirements are not directly addressed by the developed conceptual framework and instead need to be taken into account when implementing it\addtocontents{toc}{\protect\pagebreak}.